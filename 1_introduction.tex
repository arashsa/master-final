\chapter{Introduction}
\label{chap:introduction}

Semantic dependency parsing is a Natural Language Processing (NLP) task that aims at producing meaning representations at the sentence-level. We can state the aims of semantic dependency parsing as a way of expressing predicate-argument  relations in order to answer the question of \textit{Who did What to Whom?}.  In this regard there is certainly an overlap between semantic dependency parsing  and semantic role labeling, which is concerned with the task of detecting the arguments of a predicate in a sentence. Examining the research community and the publications on the field, we can see a growing interest in semantic dependency parsing in recent years. This can be attributed to both the advances that has been made in the accuracy of state-of-the-art parsers, but also due to the successful application of such parsers and their representations in a wide range of computational tasks such as Information Extraction, Textual Inference, Machine Translation, Semantic Search, and Sentiment Analysis. In this thesis we present our contribution to the field of semantic dependency parsing by building a pipeline for classifying semantic frames.

\section{Overview} 

% Background
\paragraph{Chapter 2} provides a background for our thesis. In this chapter we briefly outline dependency grammar, and then examine various approaches to dependency based parsing. We differentiate between grammar-based and data-driven dependency parsing, and we give a formal description of both approaches. Data-driven dependency parsing is itself divided in two main classes: transition-based and graph-based models. We examine these two models and give examples of their implementation by way of openly available tools. Finally, we define semantic dependency parsing, show how this approach differentiates from syntactic dependency parsing, and make that case for using semantic dependency parsing as a foundation for our thesis.

\paragraph{Chapter 3} examines a few selected state-if-the-art semantic dependency parsers. Since these parsers are all data-driven, we will also take give an overview of the various annotated corpora that have been used as basis for training these parsers.

\paragraph{Chapter 4} builds on the previous chapter by examining the results of the semantic dependency parsers that where presented there. By way of an in-depth contrastive error analysis of a set of state-of-the-art parsers we gain insights into some common errors that these parsers share, and examine if there is common ground or differences to the errors they might produce. The error analysis provides insights into the classification task that we perform in the next chapter.

\paragraph{Chapter 5} examines our own pipeline for classifying semantic frames.

\paragraph{Chapter 6} is a summary and conclusion of our thesis. We discuss the possibilities of future work that can have our thesis as its basis.