\chapter{Introduction}
\label{chap:introduction}

Semantic dependency parsing is a Natural Language Processing (NLP) task that aims at producing meaning representations at the sentence-level. We can state the aims of semantic dependency parsing as a way of expressing predicate-argument  relations in order to answer the question of \textit{Who did What to Whom}.  In this regard there is certainly an overlap between semantic dependency parsing  and semantic role labeling, which is concerned with the task of detecting the arguments of a predicate in a sentence. Examining the research community and the publications on the field, we can see a growing interest in semantic dependency parsing in recent years. This can be attributed to both the advances that has been made in the accuracy of state-of-the-art parsers, but also due to the successful application of such parsers and their representations in a wide range of computational tasks such as Information Extraction, Textual Inference, Machine Translation, Semantic Search, and Sentiment Analysis.

\section{Overview} 

% Background
\paragraph{Chapter 2} provides a background for our thesis. In this chapter we briefly  look at dependency grammar from a historical point of view, and then examine various approaches to dependency based parsing. We differentiate between grammar-based and data-driven dependency parsing. We then examine two approaches to dependency parsing: syntactic and semantic, giving a formal exposition of similarities and differences.

\paragraph{Chapter 3} examines a few selected state-if-the-art semantic dependency parsers. Since these parsers are all data-driven, we will also take a close look at the various data sets used to train these parsers.

\paragraph{Chapter 4} builds on the previous chapter by examining the results of the semantic dependency parsers that we have examined. By way of an in-depth contrastive error analysis of a set of chosen state-of-the-art parsers we gain insights into some common errors that these parsers share, but also look at what specifically differentiates their results. The error analysis provides insights that we build upon in our own work in the next chapter.

\paragraph{Chapter 5} 

\paragraph{Chapter 6} 