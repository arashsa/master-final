\chapter{Broad Coverage Semantic Dependency Parsing with Frames}
\label{chap:parsing}

% \citeauthor{Sagae:Tsuji:08} explore directed acyclic graphs (DAGs) as a way to get around the restrictions posed a tree structures. They introduce a data-driven approach to dependency parsing where DAGS are produced directly from an input string by introducing a modified version of the transition-based model described in section \ref{data-driven}. Two new transitions are introduced to the three transitions mentioned above: \textit{left-attach} and \textit{right-attach} \cite{Sagae:Tsuji:08}. These transitions can be described informally as:

% \begin{enumerate}
% \item Left-Attach: add a dependency arc $(w_i, r, w_j)$ to the set $A$ between the top two items on the stack $\alpha$, making the top item $w_i$ the head and the item below it $w_j$ the dependent, if and only if there is no arc between them already.
% \item Right-Attach: add a dependency arc $(w_i, r, w_j)$ to the set $A$ between the top two items on the stack $\alpha$, making the top item $w_i$ the dependent and the item below it $w_j$ the head, if and only if there is no arc between them already. Remove the top item on the stack $\alpha$ and place it back on the buffer $\beta$.
% \end{enumerate}